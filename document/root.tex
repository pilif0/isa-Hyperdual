\documentclass[11pt,a4paper]{article}
\usepackage{isabelle,isabellesym}

% further packages required for unusual symbols (see also
% isabellesym.sty), use only when needed

\usepackage{amsmath}
\usepackage{amssymb}
  %for \<leadsto>, \<box>, \<diamond>, \<sqsupset>, \<mho>, \<Join>,
  %\<lhd>, \<lesssim>, \<greatersim>, \<lessapprox>, \<greaterapprox>,
  %\<triangleq>, \<yen>, \<lozenge>

%\usepackage{eurosym}
  %for \<euro>

%\usepackage[only,bigsqcap]{stmaryrd}
  %for \<Sqinter>

%\usepackage{eufrak}
  %for \<AA> ... \<ZZ>, \<aa> ... \<zz> (also included in amssymb)

%\usepackage{textcomp}
  %for \<onequarter>, \<onehalf>, \<threequarters>, \<degree>, \<cent>,
  %\<currency>

% this should be the last package used
\usepackage{pdfsetup}

% urls in roman style, theory text in math-similar italics
\urlstyle{rm}
\isabellestyle{it}

% for uniform font size
%\renewcommand{\isastyle}{\isastyleminor}


\begin{document}

\title{Second-Order Hyperdual Numbers}
\author{Filip Smola, Jacques Fleuriot}
\maketitle

\begin{abstract}
  Hyperdual numbers are ones with a real component and a number of infinitesimal components, usually written as $a_0 + a_1 \cdot \epsilon_1 + a_2 \cdot \epsilon_2 + a_3 \cdot \epsilon_1\epsilon_2$.
  They have been proposed by Fike and Alonso~\cite{fike_alonso-2011} in an approach to automatic differentiation.

  In this entry we formalise second-order hyperdual numbers and this application.
  We show them to be an instance of multiple algebraic structures and then, along with facts about twice-differentiability, we define what we call the hyperdual extensions of functions on real-normed fields.
  This extension formally represents the proposed way that the first and second derivatives of a function can be automatically calculated.
  We demonstrate it on the standard logistic function $f(x) = \frac{1}{1 + e^{-x}}$ and also reproduce the example analytic function $f(x) = \frac{e^x}{\sqrt{sin(x)^3 + cos(x)^3}}$ used for demonstration by Fike and Alonso.
\end{abstract}

\tableofcontents

% sane default for proof documents
\parindent 0pt\parskip 0.5ex

% generated text of all theories
\input{session}

% optional bibliography
\bibliographystyle{abbrv}
\bibliography{root}

\end{document}

%%% Local Variables:
%%% mode: latex
%%% TeX-master: t
%%% End:
